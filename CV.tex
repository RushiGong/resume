\documentclass[10pt,letterpaper]{article}
\usepackage[letterpaper,margin=0.55in, top = 0.3in, bottom = 0.3in]{geometry}
\usepackage[utf8]{inputenc}
\usepackage{mdwlist}
\usepackage[T1]{fontenc}
\usepackage{textcomp}
\usepackage{amsmath}
\usepackage{etaremune}
\usepackage{tgpagella}
\usepackage[colorlinks=true, urlcolor=blue]{hyperref}
\pagestyle{empty}
\setlength{\tabcolsep}{0em}

% indent section style, used for sections that aren't already in lists
% that need indentation to the level of all text in the document
\newenvironment{indentsection}[1]%
{\begin{list}{}%
  {\setlength{\leftmargin}{#1}}%
  \item[]%
}
{\end{list}}

% opposite of above; bump a section back toward the left margin
\newenvironment{unindentsection}[1]%
{\begin{list}{}%
  {\setlength{\leftmargin}{-0.5#1}}%
  \item[]%
}
{\end{list}}

% format two pieces of text, one left aligned and one right aligned
\newcommand{\headerrow}[2]
%{\begin{tabular*}{\linewidth}{l@{\extracolsep{\fill}}r}
{\begin{tabular*}{\linewidth}{l@{\extracolsep{\fill}}r}
  #1 &
  #2 \\
\end{tabular*}}

% make "C++" look pretty when used in text by touching up the plus signs
\newcommand{\CPP}
{C\nolinebreak[4]\hspace{-.05em}\raisebox{.22ex}{\footnotesize\bf ++}}

%%%%%%%%%%%%%%%%%%%%%%%%%%%
%%% CONFIGURATION FLAGS %%%
%%%%%%%%%%%%%%%%%%%%%%%%%%%

% Boolean flags for which type of conference presentations are displayed

% and the actual content starts here
\begin{document}
\begin{center}
{\LARGE \textbf{Rushi Gong}}

%ADDRESS LINE 1\ \ \textbullet
%\ \ ADDRESS LINE 2\ \ \textbullet
%\ \ CITY, ST #ZIP#
%\\


(814) 441-2268\ \ \textbullet
\ \ \href{mailto:rsgong@psu.edu}{rsgong@psu.edu}
\end{center}


\hrule
\vspace{-0.6em}
\subsection*{Education}

\begin{itemize}
  \parskip=0.1em

  \item
  \headerrow
    {\textbf{The Pennsylvania State University}}
    {\textbf{University Park, PA}}
  \\
  \headerrow
    {\emph{Ph.D. Materials Science and Engineering}}
    {\emph{2020 -- 2024}}
  \begin{itemize*}
    \item 4.0 GPA
  \end{itemize*}


  \item
  \headerrow
    {\textbf{Beihang University (Beijing University of Aeronautics \& Astronautics)}}
    {\textbf{Beijing, China}}
  \\
  \headerrow
    {\emph{B.S. Materials Science and Engineering; Minor, Mathematics}}
    {\emph{2015 -- 2019}}
  \begin{itemize*}
    \item 3.76 GPA, Merit Student (Top 4\%)
  \end{itemize*}

\end{itemize}

%% TODO: add core competencies according to JD
%\hrule
%\vspace{-0.6em}
%\subsection*{Core Competencies}
%\begin{itemize}
%  \parskip=0.1em

%  \item \textbf{Alloy systems}: Pd, Zn
%  \item \textbf{Computational alloy design and development}
%\end{itemize}

\hrule
\vspace{-0.6em}
\subsection*{Research Experience}

\renewcommand\labelitemiii{$\circ$}
\begin{itemize}
  \parskip=0.1em

  \item
  \headerrow
    {\textbf{Phases Research Lab, The Pennsylvania State University}}
    {\textbf{University Park, PA}}
  \\
  \headerrow
    {\emph{Graduate Research Assistant (Advisor: Prof. Zi-Kui Liu)}}
    {\emph{2020 -- Present}}
  \begin{itemize*}
    \item Developed the Pd-Zn-based alloy thermodynamic databases and quantified uncertainty for accurate nuclearity design in catalysts
    \item First to apply Bayesian statistics and model selection for selecting liquid solution models in CALPHAD modeling
    \item Improved functionalities of computational thermodynamics tools PyCalphad and ESPEI for molten salts properties prediction
    \item Developed DFTTK structure builders to automatically generate structures for high-throughput computations
  \end{itemize*}
  \item
  \headerrow
    {\textbf{The Nuclear Science and Engineering Division, Argonne National Laboratory}}
    {\textbf{Lemont, IL}}
  \\
  \headerrow
    {\emph{Research Aide Technical PhD (Advisor: Dr. Shayan Shahbazi)}}
    {\emph{07/2022 -- 09/2022, 05/2023 -- 08/2023}}
  \begin{itemize*}
    \item Developed and quantified uncertainty of LiF-LnF3 thermodynamic databases to predict vapor-liquid equilibrium properties for distillation system design in Molten Salt Reactor
    \item Developed frameworks for calculating the Ellingham diagram to quantify redox potential of metals in fluoride molten salts
  \end{itemize*}
  \item
  \headerrow
    {\textbf{International Research Institute for Multidisciplinary Science, Beihang University}}
    {\textbf{Beijing, China}}
  \\
  \headerrow
    {\emph{Undergraduate Research Assistant (Advisor: Prof. Qianfan Zhang)}}
    {\emph{2017 -- 2019}}
  \begin{itemize*}
    \item Designed performance analysis via computational methods to examine the thermodynamics and kinetics of hydrogen evolution reaction on the transition metal adsorbed on two-dimensional catalytic substrates
  \end{itemize*}

  \item
  \headerrow
    {\textbf{Department of Materials Science and Engineering, Rensselaer Polytechnic Institute}}
    {\textbf{Troy, NY}}
  \\
  \headerrow
    {\emph{Undergraduate Research Assistant (Advisor: Prof. Yunfeng Shi)}}
    {\emph{2018}}
  \begin{itemize*}
    \item Investigated the corrosion process of the glass nanowire by using Molecular Dynamics simulations, quantified the corrosion rate and analyzed the relationship between pre-tension and corrosion
  \end{itemize*}

\end{itemize}

\hrule
\vspace{-0.6em}
\subsection*{Technical Skills}

\begin{indentsection}{\parindent}
\hyphenpenalty=1000
\begin{description*}
  \item[Computational Tools and Software:]
  Python, VASP, Thermo-Calc, Matlab, ATAT, PyCalphad, ESPEI, MongoDB
  \item [Software Developing:]
  PyCalphad, ESPEI, DFTTK (\href{https://github.com/PhasesResearchLab/DFTTK}{github.com/phasesresearchlab/dfttk})
\end{description*}
\end{indentsection}

\hrule
\vspace{-0.6em}
\subsection*{Teaching Experience}

\renewcommand\labelitemiii{$\circ$}
\begin{itemize}
    \parskip=0.1em

    \item
    \headerrow
    {\textbf{Department of Materials Science and Engineering, Penn State University}}
    {\textbf{University Park, PA}}
    \\
    \headerrow
    {\emph{Teaching Assistant}}
    {\emph{2021 -- Present}}
    \begin{itemize*}
        \item \emph{(Fall 2023)} MatSE 580: Computational Thermodynamics
        \item \emph{(Fall 2022)} MatSE 581: Computational Kinetics
        \item \emph{(Spring 2021, Spring 2023)} MatSE 410: Phase Relations in Materials Systems
    \end{itemize*}
\end{itemize}

\hrule
\vspace{-0.6em}
\subsection*{Awards}

\begin{itemize}
  \parskip=0.1em
  \item  
  \headerrow
    {MATSE Travel Award for MS\&T 23 Conference, Pennsylvania State University} {\emph{2023}}
  \item
  \headerrow
    {NSF Scholarship, Calphad L Conference} {\emph{2023}}
  \item  
  \headerrow
    {Thrower Travel Award for Calphad L Conference, Pennsylvania State University} {\emph{2023}}
\end{itemize}


\hrule
\vspace{-0.6em}
\subsection*{Publications}

% Assumes that the ENUMERATE_PUBLICATIONS flag exists
%!TEX root = cv.tex
% TODO: Try to use bibliographic files here. Some WIP in min/minimal.tex
% TODO: Have some kind of highlighted flagging system where a short list could be generated with only key highlighted references.

\begin{etaremune}

% Formatting:
% \item <Authors>
% <Title>
% <Publication>
% <Link>
\item J.P.S. Palma, \textbf{R. Gong}, B.J. Bocklund, R. Otis, M. Poschmann, M. Piro, Y. Wang, S. Shahbazi, T.G. Levitskaia, S. Hu,  N.D. Smith, H. Kim, Z.K. Liu, and S.L. Shang, 
Thermodynamic modeling with uncertainty quantification using the modified quasichemical model in quadruplet approximation: Implementation into PyCalphad and ESPEI,  
\href{http://arxiv.org/abs/2204.09111}{arxiv.org/abs/2204.09111}.
\item H. Sun, S.L. Shang, \textbf{R. Gong}, B.J. Bocklund, A.M. Beese, Z.K. Liu, 
Thermodynamic modeling of the Nb-Ni system with uncertainty quantification using PyCalphad and ESPEI, 
\textbf{Calphad}, 2023,
\href{https://doi.org/10.1016/j.calphad.2023.102563}{doi.org/10.1016/j.calphad.2023.102563}.
\item \textbf{R. Gong}, S.L. Shang, H. Sun, M.J. Janik, and Z.K. Liu,
Thermodynamic modeling of the Pd-Zn system with uncertainty quantification and its implication to tailor catalysts, 
\textbf{Calphad}, 2022,
\href{https://doi.org/10.1016/j.calphad.2022.102491}{doi.org/10.1016/j.calphad.2022.102491}.
\item  A. Dasgupta, H. He, \textbf{R. Gong}, S.L. Shang, E.K. Zimmerer, R.J. Meyer, Z.K. Liu, M.J. Janik, and R.M. Rioux,
Atomic control of active site ensembles in ordered alloys to enhance hydrogenation selectivity, 
\textbf{Nature Chemistry}, 14, 523–529 (2022),
\href{https://doi.org/10.1038/s41557-021-00855-3}{doi: 10.1038/s41557-021-00855-3}.

\end{etaremune}

\hrule
\vspace{-0.6em}
\subsection*{Presentations}

% Assumes that the ENUMERATE_PUBLICATIONS flag exists
%!TEX root = cv.tex
% TODO: Try to use bibliographic files here. Some WIP in min/minimal.tex
% TODO: Have some kind of highlighted flagging system where a short list could be generated with only key highlighted references.

\begin{etaremune}

% Formatting:
% <Authors>
% <(Year, Month)>
% <Title>
% <Name of conference>,
% <Location>
\item \textbf{R. Gong*}, S.L. Shang, V. Goncharov, B. Merrill, X. Guo, Z.K. Liu
(2022, October) \emph{Invited}.
Exploring and Implementing Thermodynamic Models for Liquid and their Applications to Thermodynamic Modeling of Molten Salts. 
Materials Science and Technology 2023,
Columbus, OH.
\item \textbf{R. Gong*}, S. Shahbazi
(2023, July).
Thermodynamic Modeling and Model Selection for LiF-LnF3 Molten Salts with Uncertainty Propagation.
Molten Salt Thermal Properties Uncertainty Workshop,
Lemont, IL.
\item \textbf{R. Gong*}, S.L. Shang, G. Canning, R.M. Rioux, M.J. Janik, Z.K. Liu
(2022, October).
Thermodynamic Modeling with Uncertainty Quantification and its Implications for Intermetallic Catalysts Design: Application to PdZn-Based Gamma-Brass Phase.
Materials Science and Technology 2022,
Pittsburgh, PA.

\end{etaremune}
\vspace{-0.6em}
\hspace{1.5em}{\small \emph{\textbf{*} presenter}}
\vspace{0.4em}

\end{document}
