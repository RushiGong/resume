\documentclass[10pt,letterpaper]{article}
\usepackage[letterpaper,margin=0.55in, top = 0.3in, bottom = 0.3in]{geometry}
\usepackage[utf8]{inputenc}
\usepackage{mdwlist}
\usepackage[T1]{fontenc}
\usepackage{textcomp}
\usepackage{amsmath}
\usepackage{tgpagella}
\usepackage{etaremune}
\usepackage[colorlinks=true, urlcolor=blue]{hyperref}
\pagestyle{empty}
\setlength{\tabcolsep}{0em}

% indent section style, used for sections that aren't already in lists
% that need indentation to the level of all text in the document
\newenvironment{indentsection}[1]%
{\begin{list}{}%
  {\setlength{\leftmargin}{#1}}%
  \item[]%
}
{\end{list}}

% opposite of above; bump a section back toward the left margin
\newenvironment{unindentsection}[1]%
{\begin{list}{}%
  {\setlength{\leftmargin}{-0.5#1}}%
  \item[]%
}
{\end{list}}

% format two pieces of text, one left aligned and one right aligned
\newcommand{\headerrow}[2]
%{\begin{tabular*}{\linewidth}{l@{\extracolsep{\fill}}r}
{\begin{tabular*}{\linewidth}{l@{\extracolsep{\fill}}r}
  #1 &
  #2 \\
\end{tabular*}}

% make "C++" look pretty when used in text by touching up the plus signs
\newcommand{\CPP}
{C\nolinebreak[4]\hspace{-.05em}\raisebox{.22ex}{\footnotesize\bf ++}}

% and the actual content starts here
\begin{document}
\begin{center}
{\LARGE \textbf{Rushi Gong}}

%ADDRESS LINE 1\ \ \textbullet
%\ \ ADDRESS LINE 2\ \ \textbullet
%\ \ CITY, ST #ZIP#
%\\


(814) 441-2268\ \ \textbullet
\ \ \href{mailto:rsgong@psu.edu}{rsgong@psu.edu}
\end{center}


\hrule
\vspace{-0.6em}
\subsection*{Education}

\begin{itemize}
  \parskip=0.1em

  \item
  \headerrow
    {\textbf{The Pennsylvania State University}}
    {\textbf{University Park, PA}}
  \\
  \headerrow
    {\emph{Ph.D. Materials Science and Engineering}}
    {\emph{2020 -- 2024}}
  \begin{itemize*}
    \item 4.0 GPA
  \end{itemize*}


  \item
  \headerrow
    {\textbf{Beihang University (Beijing University of Aeronautics \& Astronautics)}}
    {\textbf{Beijing, China}}
  \\
  \headerrow
    {\emph{B.S. Materials Science and Engineering; Minor, Mathematics}}
    {\emph{2015 -- 2019}}
  \begin{itemize*}
    \item 3.76 GPA, Merit Student (Top 4\%)
  \end{itemize*}

\end{itemize}


\hrule
\vspace{-0.6em}
\subsection*{Research Experience}

\renewcommand\labelitemiii{$\circ$}
\begin{itemize}
  \parskip=0.1em

  \item
  \headerrow
    {\textbf{Phases Research Lab, The Pennsylvania State University}}
    {\textbf{University Park, PA}}
  \\
  \headerrow
    {\emph{Graduate Research Assistant (Advisor: Prof. Zi-Kui Liu)}}
    {\emph{2020 -- Present}}
  \begin{itemize*}
    \item Developed the Pd-Zn-based alloy thermodynamic databases and quantified uncertainty for accurate nuclearity design for catalysts
    \item First to apply Bayesian statistics and model selection in selecting liquid solution models for CALPHAD modeling
    \item Developed DFTTK structure builders to automatically generate structures for high-throughput computations
  \end{itemize*}
  \item
  \headerrow
    {\textbf{The Nuclear Science and Engineering Division, Argonne National Laboratory}}
    {\textbf{Lemont, IL}}
  \\
  \headerrow
    {\emph{Research Aide Technical PhD (Advisor: Dr. Shayan Shahbazi)}}
    {\emph{07/2022 -- 09/2022, 05/2023 -- 08/2023}}
  \begin{itemize*}
    \item Developed and quantified uncertainty of LiF-LnF3 thermodynamic databases to predict vapor-liquid equilibrium properties
    \item Developed frameworks to calculate the Ellingham diagram for quantifying redox potential of metals fluorides
  \end{itemize*}
  \item
  \headerrow
    {\textbf{International Research Institute for Multidisciplinary Science , Beihang University}}
    {\textbf{Beijing, China}}
  \\
  \headerrow
    {\emph{Undergraduate Research Assistant (Advisor: Prof. Qianfan Zhang)}}
    {\emph{2017 -- 2019}}
  \begin{itemize*}
    \item Designed performance analysis via computational methods to examine the thermodynamics and kinetics of hydrogen evolution reaction on the transition metal adsorbed on two-dimensional catalytic substrates
  \end{itemize*}

  \item
  \headerrow
    {\textbf{Department of Materials Science and Engineering, Rensselaer Polytechnic Institute}}
    {\textbf{Troy, NY}}
  \\
  \headerrow
    {\emph{Undergraduate Research Assistant (Advisor: Prof. Yunfeng Shi)}}
    {\emph{2018}}
  \begin{itemize*}
    \item Investigated the corrosion process of the glass nanowire by using Molecular Dynamics simulations, quantified the corrosion rate and analyzed the relationship between pre-tension and corrosion
  \end{itemize*}

\end{itemize}

\hrule
\vspace{-0.6em}
\subsection*{Technical Skills}

\begin{indentsection}{\parindent}
\hyphenpenalty=1000
\begin{description*}
  \item[Computational Tools and Software:]
  Python, VASP, Thermo-Calc, Matlab, ATAT, PyCalphad, ESPEI, MongoDB
  \item [Software Developing:]
  DFTTK (\href{https://github.com/PhasesResearchLab/DFTTK}{github.com/phasesresearchlab/dfttk})
\end{description*}
\end{indentsection}

\hrule
\vspace{-0.6em}
\subsection*{Publications}

% Assumes that the ENUMERATE_PUBLICATIONS flag exists
%!TEX root = cv.tex
% TODO: Try to use bibliographic files here. Some WIP in min/minimal.tex
% TODO: Have some kind of highlighted flagging system where a short list could be generated with only key highlighted references.

\begin{etaremune}

% Formatting:
% \item <Authors>
% <Title>
% <Publication>
% <Link>

%NEAMS activities
\item S. Shahbazi, M. Tano, S. Thomas, S. Walker, A.A. Jaoude, Y. Jeong, \textbf{R. Gong}, D.H. Kam, B. Chen, D. Grabaskas,
NEAMS activities supporting mechanistic source term model development for molten salt reactors,
\textbf{PSA conference}, 2023,
\href{https://doi.org/10.13182/PSA23-41261}
{doi: 10.13182/PSA23-41261}.
%MQMQA demo
\item J.P.S. Palma, \textbf{R. Gong}, B.J. Bocklund, R. Otis, M. Poschmann, M. Piro, S. Shahbazi, T.G. Levitskaia, S. Hu,  N.D. Smith, Y. Wang, H. Kim, Z.K. Liu, and S.L. Shang, 
Thermodynamic modeling with uncertainty quantification using the modified quasichemical model in quadruplet approximation: Implementation into PyCalphad and ESPEI,  
\textbf{Calphad}, 2023,
\href{https://doi.org/10.1016/j.calphad.2023.102618}
{doi: 10.1016/j.calphad.2023.102618}.
%Nb-Ni Modeling
\item H. Sun, S.L. Shang, \textbf{R. Gong}, B.J. Bocklund, A.M. Beese, Z.K. Liu, 
Thermodynamic modeling of the Nb-Ni system with uncertainty quantification using PyCalphad and ESPEI, 
\textbf{Calphad}, 2023,
\href{https://doi.org/10.1016/j.calphad.2023.102563}{doi: 10.1016/j.calphad.2023.102563}.
%Pd-Zn Modeling
\item \textbf{R. Gong}, S.L. Shang, H. Sun, M.J. Janik, and Z.K. Liu,
Thermodynamic modeling of the Pd-Zn system with uncertainty quantification and its implication to tailor catalysts, 
\textbf{Calphad}, 2022,
\href{https://doi.org/10.1016/j.calphad.2022.102491}{doi: 10.1016/j.calphad.2022.102491}.
%Active site design
\item  A. Dasgupta, H. He, \textbf{R. Gong}, S.L. Shang, E.K. Zimmerer, R.J. Meyer, Z.K. Liu, M.J. Janik, and R.M. Rioux,
Atomic control of active site ensembles in ordered alloys to enhance hydrogenation selectivity, 
\textbf{Nature Chemistry}, 14, 523–529 (2022),
\href{https://doi.org/10.1038/s41557-021-00855-3}{doi: 10.1038/s41557-021-00855-3}.

\end{etaremune}

\hrule
\vspace{-0.6em}
\subsection*{Teaching Experience}

\renewcommand\labelitemiii{$\circ$}
\begin{itemize}
    \parskip=0.1em

    \item
    \headerrow
    {\textbf{Department of Materials Science and Engineering, Penn State University}}
    {\textbf{University Park, PA}}
    \\
    \headerrow
    {\emph{Teaching Assistant}}
    {\emph{2021}}
    \begin{itemize*}
        \item MatSE 410: Phase Relations in Materials Systems
    \end{itemize*}
\end{itemize}

\end{document}
