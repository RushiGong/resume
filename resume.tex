\documentclass[10pt,letterpaper]{article}
\usepackage[letterpaper,margin=0.55in, top = 0.3in, bottom = 0.3in]{geometry}
\usepackage[utf8]{inputenc}
\usepackage{mdwlist}
\usepackage[T1]{fontenc}
\usepackage{textcomp}
\usepackage{amsmath}
\usepackage{tgpagella}
\usepackage{etaremune}
\usepackage[colorlinks=true, urlcolor=blue]{hyperref}
\pagestyle{empty}
\setlength{\tabcolsep}{0em}

% indent section style, used for sections that aren't already in lists
% that need indentation to the level of all text in the document
\newenvironment{indentsection}[1]%
{\begin{list}{}%
  {\setlength{\leftmargin}{#1}}%
  \item[]%
}
{\end{list}}

% opposite of above; bump a section back toward the left margin
\newenvironment{unindentsection}[1]%
{\begin{list}{}%
  {\setlength{\leftmargin}{-0.5#1}}%
  \item[]%
}
{\end{list}}

% format two pieces of text, one left aligned and one right aligned
\newcommand{\headerrow}[2]
%{\begin{tabular*}{\linewidth}{l@{\extracolsep{\fill}}r}
{\begin{tabular*}{\linewidth}{l@{\extracolsep{\fill}}r}
  #1 &
  #2 \\
\end{tabular*}}

% make "C++" look pretty when used in text by touching up the plus signs
\newcommand{\CPP}
{C\nolinebreak[4]\hspace{-.05em}\raisebox{.22ex}{\footnotesize\bf ++}}

% and the actual content starts here
\begin{document}
\begin{center}
{\LARGE \textbf{Rushi Gong}}

%ADDRESS LINE 1\ \ \textbullet
%\ \ ADDRESS LINE 2\ \ \textbullet
%\ \ CITY, ST #ZIP#
%\\


(814) 441-2268\ \ \textbullet
\ \ \href{mailto:rsgong@psu.edu}{rsgong@psu.edu}
\end{center}


\hrule
\vspace{-0.6em}
\subsection*{Research Experience}

\renewcommand\labelitemiii{$\circ$}
\begin{itemize}
  \parskip=0.1em

  \item
  \headerrow
    {\textbf{Phases Research Lab, The Pennsylvania State University}}
    {\textbf{University Park, PA}}
  \\
  \headerrow
    {\emph{Graduate Research Assistant (Advisor: Prof. Zi-Kui Liu)}}
    {\emph{2020 -- Present}}
  \begin{itemize*}
    \item Developed Pd-Zn-based gamma-brass alloys using DFT-based first-principles calculations, machine learning, and CALPHAD modeling to achieve atomic control of catalytic active-site ensembles and enhance the hydrogenation selectivity of catalysts.
    \item Developed thermodynamic databases with uncertainty quantification for molten salts including Cr and Ni in FLiNaK and LaCl3 in LiCl-KCl through high-throughput computational thermodynamics to improve predictive accuracy of thermochemical properties and critical characteristics of molten salts solids and liquids.
    \item Applied Bayesian statistics and model selections in conducting a statistical comparison of liquid solution models in CALPHAD modeling, effectively quantifying the uncertainty for the selected models through Markov Chain Monte Carlo.
    \item Designed a framework for integrating custom models into the open-source software PyCalphad, and led the implementation of the universal quasichemical model and molecular interaction volume model, broadening the applicability of models.
    \item Established the structure builder in the open-source software DFTTK to automatically generate structures for high-throughput DFT calculations and CALPHAD modeling.
  \end{itemize*}
  \item
  \headerrow
    {\textbf{The Nuclear Science and Engineering Division, Argonne National Laboratory}}
    {\textbf{Lemont, IL}}
  \\
  \headerrow
    {\emph{Research Aide Technical PhD (Advisor: Dr. Shayan Shahbazi)}}
    {\emph{07/2022 -- 09/2022, 05/2023 -- 08/2023}}
  \begin{itemize*}
    \item Developed and quantified uncertainty of LiF-LnF3 thermodynamic databases to facilitate accurate predictions of vapor-liquid equilibrium properties for designing distillation systems in Molten Salt Reactors.
    \item Enhanced and upgraded functionalities of computational thermodynamics tools PyCalphad and ESPEI to predict thermodynamic properties of molten salts. Implemented features such as the Ellingham diagram calculations, redox potential calculations for corrosion properties, vapor pressure calculations, and improved calculation stability on multi-component systems. 
  \end{itemize*}
  \item
  \headerrow
    {\textbf{International Research Institute for Multidisciplinary Science, Beihang University}}
    {\textbf{Beijing, China}}
  \\
  \headerrow
    {\emph{Undergraduate Research Assistant (Advisor: Prof. Qianfan Zhang)}}
    {\emph{2017 -- 2019}}
  \begin{itemize*}
    \item Investigated electronic structure of two-dimensional material FeOCl through DFT calculations, applied strain and doping to control magnetic properties.
    \item Developed a performance analysis framework using DFT calculations to investigate the thermodynamics and kinetics of the hydrogen evolution reaction on transition metals adsorbed on two-dimensional catalytic substrates.
  \end{itemize*}

  \item
  \headerrow
    {\textbf{Department of Materials Science and Engineering, Rensselaer Polytechnic Institute}}
    {\textbf{Troy, NY}}
  \\
  \headerrow
    {\emph{Undergraduate Research Assistant (Advisor: Prof. Yunfeng Shi)}}
    {\emph{2018}}
  \begin{itemize*}
    \item Investigated the corrosion process of the glass nanowire by using Molecular Dynamics simulations, quantified the corrosion rate, and analyzed the relationship between pre-tension and corrosion.
  \end{itemize*}

\end{itemize}

\hrule
\vspace{-0.6em}
\subsection*{Technical Skills}

\begin{indentsection}{\parindent}
\hyphenpenalty=1000
\begin{description*}
  \item[Computational approaches:]
  CALPHAD, DFT, Ab initio Molecular Dynamics (AIMD), Machine learning
  \item[Computational languages and tools:]
  Python, Linux, Matlab, XML, VASP, Thermo-Calc, ATAT, MongoDB, LAMMPS
  \item [Software developing:]
  PyCalphad, ESPEI, and DFTTK. (\href{https://github.com/RushiGong}{github.com/RushiGong})
\end{description*}
\end{indentsection}

\hrule
\vspace{-0.6em}
\subsection*{Education}

\begin{itemize}
  \parskip=0.1em

  \item
  \headerrow
    {\textbf{The Pennsylvania State University}}
    {\textbf{University Park, PA}}
  \\
  \headerrow
    {\emph{Ph.D. in Materials Science and Engineering}}
    {\emph{2020 -- 2024}}
  \begin{itemize*}
    \item GPA: 4.0/4.0
  \end{itemize*}


  \item
  \headerrow
    {\textbf{Beihang University}}
    {\textbf{Beijing, China}}
  \\
  \headerrow
    {\emph{B.S. in Materials Science and Engineering; Minor, Mathematics}}
    {\emph{2015 -- 2019}}
  \begin{itemize*}
    \item GPA: 3.76/4.0, Merit Student (Top 4\%)
  \end{itemize*}

\end{itemize}

\end{document}
