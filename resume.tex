\documentclass[10pt,letterpaper]{article}
\usepackage[letterpaper,margin=0.55in, top = 0.3in, bottom = 0.3in]{geometry}
\usepackage[utf8]{inputenc}
\usepackage{mdwlist}
\usepackage[T1]{fontenc}
\usepackage{textcomp}
\usepackage{amsmath}
\usepackage{tgpagella}
\usepackage{etaremune}
\usepackage[colorlinks=true, urlcolor=blue]{hyperref}
\pagestyle{empty}
\setlength{\tabcolsep}{0em}

% indent section style, used for sections that aren't already in lists
% that need indentation to the level of all text in the document
\newenvironment{indentsection}[1]%
{\begin{list}{}%
  {\setlength{\leftmargin}{#1}}%
  \item[]%
}
{\end{list}}

% opposite of above; bump a section back toward the left margin
\newenvironment{unindentsection}[1]%
{\begin{list}{}%
  {\setlength{\leftmargin}{-0.5#1}}%
  \item[]%
}
{\end{list}}

% format two pieces of text, one left aligned and one right aligned
\newcommand{\headerrow}[2]
%{\begin{tabular*}{\linewidth}{l@{\extracolsep{\fill}}r}
{\begin{tabular*}{\linewidth}{l@{\extracolsep{\fill}}r}
  #1 &
  #2 \\
\end{tabular*}}

% make "C++" look pretty when used in text by touching up the plus signs
\newcommand{\CPP}
{C\nolinebreak[4]\hspace{-.05em}\raisebox{.22ex}{\footnotesize\bf ++}}

% and the actual content starts here
\begin{document}
\begin{center}
{\LARGE \textbf{Rushi Gong}}

%ADDRESS LINE 1\ \ \textbullet
%\ \ ADDRESS LINE 2\ \ \textbullet
%\ \ CITY, ST #ZIP#
%\\


(814) 441-2268\ \ \textbullet
\ \ \href{mailto:rsgong@psu.edu}{rsgong@psu.edu}
\end{center}


\hrule
\vspace{-0.6em}
\subsection*{Education}

\begin{itemize}
  \parskip=0.1em

  \item
  \headerrow
    {\textbf{The Pennsylvania State University}}
    {\textbf{University Park, PA}}
  \\
  \headerrow
    {\emph{Ph.D. Materials Science and Engineering}}
    {\emph{2020 -- 2024}}
  \begin{itemize*}
    \item 4.0 GPA
  \end{itemize*}


  \item
  \headerrow
    {\textbf{Beihang University (Beijing University of Aeronautics \& Astronautics)}}
    {\textbf{Beijing, China}}
  \\
  \headerrow
    {\emph{B.S. Materials Science and Engineering; Minor, Mathematics}}
    {\emph{2015 -- 2019}}
  \begin{itemize*}
    \item 3.76 GPA
    \item Merit Student (Top 4\%)
  \end{itemize*}

\end{itemize}


\hrule
\vspace{-0.6em}
\subsection*{Research Experience}

\renewcommand\labelitemiii{$\circ$}
\begin{itemize}
  \parskip=0.1em

  \item
  \headerrow
    {\textbf{Phases Research Lab, The Pennsylvania State University}}
    {\textbf{University Park, PA}}
  \\
  \headerrow
    {\emph{Graduate Research Assistant (Advisor: Prof. Zi-Kui Liu)}}
    {\emph{2020 -- Present}}
  \begin{itemize*}
    \item First to apply thermodynamic modeling in the investigation of site nuclearity on Pd-Zn-based catalysts surfaces
    \item Developed the Pd-Zn-based alloy thermodynamic databases and quantified uncertainty by leveraging the distribution of model parameters during optimization for accurate nuclearity design
    \item Built framework to drive selections for stable intermetallic catalysts candidates with Machine Learning and first-principles calculations tools
    \item Developed DFTTK structure builders to automatically generate structures for high-throughput computations
  \end{itemize*}
  \item
  \headerrow
    {\textbf{The Nuclear Science and Engineering Division, Argonne National Laboratory}}
    {\textbf{Lemont, IL}}
  \\
  \headerrow
    {\emph{Research Aide Technical PhD (Advisor: Dr. Shayan Shahbazi)}}
    {\emph{07/2022 -- 09/2022}}
  \begin{itemize*}
    \item Developed LiF-LnF3 thermodynamic databases to predict relative volatility of Ln
    \item Quantified uncertainty and sensitivity of thermodynamic modeling of fluoride molten salts
  \end{itemize*}
  \item
  \headerrow
    {\textbf{International Research Institute for Multidisciplinary Science , Beihang University}}
    {\textbf{Beijing, China}}
  \\
  \headerrow
    {\emph{Undergraduate Research Assistant (Advisor: Prof. Qianfan Zhang)}}
    {\emph{2017 -- 2019}}
  \begin{itemize*}
    \item Built stable substrates with transition metal adsorbed on two-dimensional materials as catalysts
    \item Designed performance analysis via computational methods to examine the thermodynamics and kinetics of hydrogen evolution reaction on the catalytic substrates
  \end{itemize*}

  \item
  \headerrow
    {\textbf{Department of Materials Science and Engineering, Rensselaer Polytechnic Institute}}
    {\textbf{Troy, NY}}
  \\
  \headerrow
    {\emph{Undergraduate Research Assistant (Advisor: Prof. Yunfeng Shi)}}
    {\emph{2018}}
  \begin{itemize*}
    \item Investigated the corrosion process of the glass nanowire by using Molecular Dynamics simulations
    \item Quantified the corrosion rate and analyzied the relationship between pre-tension and corrosion
  \end{itemize*}

\end{itemize}

\hrule
\vspace{-0.6em}
\subsection*{Teaching Experience}

\renewcommand\labelitemiii{$\circ$}
\begin{itemize}
    \parskip=0.1em

    \item
    \headerrow
    {\textbf{Department of Materials Science and Engineering, Penn State University}}
    {\textbf{University Park, PA}}
    \\
    \headerrow
    {\emph{Teaching Assistant}}
    {\emph{2021}}
    \begin{itemize*}
        \item MatSE 410: Phase Relations in Materials Systems
    \end{itemize*}
\end{itemize}



\hrule
\vspace{-0.6em}
\subsection*{Technical Skills}

\begin{indentsection}{\parindent}
\hyphenpenalty=1000
\begin{description*}
  \item[Computational Tools and Software:]
  Python, MongoDB, VASP, Thermo-Calc, Matlab, ATAT, PyCalphad, ESPEI
  \item [Software Developing:]
  DFTTK (\href{https://github.com/PhasesResearchLab/DFTTK}{github.com/phasesresearchlab/dfttk})
\end{description*}
\end{indentsection}

\hrule
\vspace{-0.6em}
\subsection*{Publications}

\hyphenpenalty=1000
\begin{etaremune}
  \item \textbf{R. Gong}, S. L. Shang, H. Sun, M. J. Janik, and Z. K. Liu,
  Thermodynamic modeling of the Pd-Zn system with uncertainty quantification and its implication to tailor catalysts, 
  \textbf{Calphad}, 2022, \href{https://doi.org/10.1016/j.calphad.2022.102491}{doi.org/10.1016/j.calphad.2022.102491}.
  \item  A. Dasgupta, H. He, \textbf{R. Gong}, S. L. Shang, E. K. Zimmerer, R. J. Meyer, Z. K. Liu, M. J. Janik, and R. M. Rioux,
  Atomic control of active site ensembles in ordered alloys to enhance hydrogenation selectivity, 
  \textbf{Nature Chemistry}, 14, 523–529 (2022), \href{https://doi.org/10.1038/s41557-021-00855-3}{doi: 10.1038/s41557-021-00855-3}.
  \item J. P. S. Palma, \textbf{R. Gong}, B. J. Bocklund, R. Otis, M. Poschmann, M. Piro, Y. Wang, T. G. Levitskaia, S. Hu,  N. D. Smith, H. Kim, Z. K. Liu, and S. L. Shang, 
  Thermodynamic modeling with uncertainty quantification using the modified quasichemical model in quadruplet approximation: Implementation into PyCalphad and ESPEI,  \href{http://arxiv.org/abs/2204.09111}{arxiv.org/abs/2204.09111}.
\end{etaremune}


\end{document}
